\documentclass[a4paper, 10pt]{article}

\usepackage[russian]{babel}
\usepackage[T2A]{fontenc}
\usepackage[utf8]{inputenc}
\usepackage{multicol}
\usepackage{adjustbox}
\usepackage{amsmath}
\usepackage{tikz}
\usepackage{amssymb}
\usepackage{setspace}



\usepackage{geometry}
\geometry{top=10mm}
\geometry{bottom=5mm}
\geometry{left=10mm}
\geometry{right=10mm}

\usepackage{graphicx}
\usepackage{wrapfig}

\pagestyle{empty}

\definecolor{myorange}{HTML}{d94d1a}
\definecolor{myyellow}{HTML}{f2b31a}

\newcommand{\solidline}[6]{
    \begin{tikzpicture}[scale=1.5, line width=2pt]
        \draw[#6, #3] (#4,1) -- (#5,1);
        
        \node[above] at (#4,1) {\tiny #1};
        \node[above] at (#5,1) {\tiny #2};
    \end{tikzpicture}
}

\newcommand{\longline}[3]{
    \solidline{#1}{#2}{#3}{1}{1.6}{}
}

\newcommand{\shortline}[3]{
    \solidline{#1}{#2}{#3}{1}{1.4}{}
}

\newcommand{\shortdashedline}[3]{
    \solidline{#1}{#2}{#3}{1}{1.4}{dashed}
}

\newcommand{\splitline}[2]{
    \begin{tikzpicture}[scale=1.5, line width=2pt]
        \draw[black] (1,1) -- (1.3,1);
        \draw[dashed, black] (1.3,1) -- (1.6,1);
        \node[above] at (1,1) {\tiny #1};
        \node[above] at (1.6,1) {\tiny #2};
    \end{tikzpicture}
}



\begin{document}

\begin{minipage}{0.59\textwidth}
\hfill КНИГА II ПРЕДЛ. XIII. ТЕОРЕМА \hfill {8\raisebox{-0.2em}{9}}
\vspace{3mm}
    \begin{wrapfigure}[5]{l}{0.19\textwidth}
        \includegraphics[width=0.2\textwidth]{image}
    \end{wrapfigure}
    
    \\
    \begin{spacing}{1.2}
        \large \textit{о всяком треугольнике квадрат стороны, стягивающей острый угол, меньше суммы квадратов сторон, содержащих этот угол, на дважды прямоугольник, заключенный между любой из этих сторон и отрезком, отсекеаемым перпендикуляром из противоположного угла от этого отрезка или от продленного отрезка.}
    \end{spacing}

    \noindent \large \textit{Первый случай} \\
    \longline{C}{$A^2$}{blue}$<$\splitline{B}{$C^2$}$+$\longline{A}{$B^2$}{myorange}\textit{на 2} \cdot 
    \splitline{B}{C}
    \cdot
    \shortline{B}{D}{black}
    .
    
    \noindent \textit{Второй случай}
    
    \longline{B}{$C^2$}{blue}$<$\longline{B}{$F^2$}{myorange}$+$\shortline{F}{$G^2$}{black}\textit{на 2} \cdot
    \shortline{F}{G}{black}
    \cdot
    \splitline{F}{H}
    .\\

    \begin{center}
        {\Large Предположим, перпендикуляр падает \\ внутри треугольника, тогда (пр. {\MakeUppercase{\romannumeral 2}}.\raisebox{-0.2em}{7})}

        \splitline{B}{$C^2$}$+$\shortline{B}{$D^2$}{black}
        = 2 \cdot
        \splitline{B}{C} \cdot \shortline{B}{D}{black} + \shortdashedline{D}{$C^2$}{black},
        \\
        к каждой добавим \longline{A}{$D^2$}{myyellow}, тогда
        \\
        \splitline{B}{$C^2$} + \shortline{B}{$D^2$}{black} + \longline{A}{$D^2$}{myyellow} = 
        \\
        2 \cdot \splitline{B}{C} \cdot \shortline{B}{D}{black} + \shortdashedline{D}{$C^2$}{black} + \longline{A}{$D^2$}{myyellow}
        \\
        $\therefore$(пр. I.\raisebox{-0.2em}{47})
        \\
        \splitline{B}{$C^2$} + \longline{A}{$B^2$}{myorange}$= 2$ \cdot\splitline{B}{C}\cdot\shortline{B}{D}{black}$+$\longline{C}{$A^2$}{blue},
        \\
        и $\therefore$\longline{C}{$A^2$}{blue}$<$\splitline{B}{$C^2$}$+$\longline{A}{$B^2$}{myorange} на 2 \cdot\splitline{B}{C}\cdot\shortdashedline{D}{C}{black}

    {\Large Теперь предположим, что перпендикуляр \\ падает вовне треугольника, тогда (пр. II.\raisebox{-0.2em}{7})}

    \splitline{F}{$H^2$} + \shortline{F}{$G^2$}{black} = 2 \cdot\splitline{F}{H}\cdot\shortline{F}{G}{black} + \shortdashedline{G}{$H^2$}{black},
    \\
    к каждой добавим \longline{H}{$E^2$}{myyellow}, тогда
    \\
    \splitline{F}{$H^2$} + \shortline{F}{$G^2$}{black} + \longline{H}{$E^2$}{myyellow} = 
    \\
    2 \cdot\splitline{F}{H}\cdot\shortline{F}{G}{black}$+$\shortdashedline{G}{$H^2$}{black}$+$\longline{H}{$E^2$}{myyellow}
    \\
    $\therefore$(пр. I.\raisebox{-0.2em}{47})\longline{E}{F}{myorange}$+$\shortline{F}{$G^2$}{black}= 2 \cdot\splitline{F}{H}\cdot\shortline{F}{$G^2$}{black}$+$\longline{E}{$G^2$}{blue},
    \\
    $\therefore$ \longline{E}{$G^2$}{blue}$<$\longline{E}{$F^2$}{myorange}$+$\shortline{F}{$G^2$}{black} на 2\cdot\splitline{F}{H}\cdot\shortline{F}{G}{black}.
    
    \end{center}
    
    \begin{flushright}
        ч.т.д.
    \end{flushright}
    
\end{minipage}
\hfill
\begin{minipage}{.3\textwidth}

    \begin{center}
     \Large Первый случай
    \end{center}
    \begin{tikzpicture}[scale=1.5, line width=2pt]
        \draw[myyellow] (1.1,1) -- (1.1,4);
        \draw[myorange] (0,1) -- (1.1,4);
        \draw[blue] (1.1,4) -- (3,1);
        \draw[dashed, black] (3,1) -- (1.1,1);
        \draw[black] (1.1,1) -- (0,1);
        
    
        \node[below left] at (0,1) {B};
        \node[below] at (1.1,1) {D};
        \node[above] at (1.1,4) {A};
        \node[below right] at (3,1) {C};
    \end{tikzpicture}
    
    \begin{center}
     \Large Второй случай
    \end{center}
    \begin{tikzpicture}[scale=1.5, line width=2pt]
        \draw[blue] (1.1,6.2) -- (3,9.2);
        \draw[myyellow] (3,6.2) -- (3,9.2);
        \draw[myorange] (0,6.2) -- (3,9.2);
        \draw[dashed, black] (3,6.2) -- (1.1,6.2);
        \draw[black] (1.1,6.2) -- (0,6.2);
        
    
        \node[below left] at (0,6.2) {F};
        \node[below] at (1.1,6.2) {G};
        \node[above] at (3,9.2) {E};
        \node[below right] at (3,6.2) {H};
    \end{tikzpicture}
    \vspace{40mm}
\end{minipage}	



\end{document}
