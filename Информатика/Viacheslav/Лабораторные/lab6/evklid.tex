\documentclass[a4paper, 10pt]{article}

\usepackage[russian]{babel}
\usepackage[T2A]{fontenc}
\usepackage[utf8]{inputenc}
\usepackage{multicol}
\usepackage{adjustbox}
\usepackage{amsmath}
\usepackage{tikz}
\usepackage{amssymb}



\usepackage{geometry}
\geometry{top=10mm}
\geometry{bottom=5mm}
\geometry{left=10mm}
\geometry{right=10mm}

\usepackage{graphicx}
\usepackage{wrapfig}

\pagestyle{empty}

\begin{document}
\begin{minipage}{0.55\textwidth}
\hfill КНИГА II ПРЕДЛ. XIII. ТЕОРЕМА \hfill \llap{89}
\vspace{6mm}
    \begin{wrapfigure}{l}{0.13\textwidth}
        \includegraphics[width=0.14\textwidth]{image}
    \end{wrapfigure}
    \\
    \textit{\large о всяком треугольнике квадрат стороны, стягивающей острый угол, меньше суммы квадратов сторон, содержащих этот угол, на дважды прямоугольник, заключенный между любой из этих сторон и отрезком, отсекеаемым перпендикуляром из противоположного угла от этого отрезк или от продленного отрезка.}

    \noindent \textit{Первый случай}
    
    \begin{tikzpicture}[scale=1.5, line width=2pt]
        \draw[blue] (1,1) -- (1.6,1);
        
        \node[above left] at (1.1,1) {\tiny C};
        \node[above right] at (1.5,1) {\tiny $A^2$};
    \end{tikzpicture}
    $<$
    \begin{tikzpicture}[scale=1.5, line width=2pt]
        \draw[black] (1,1) -- (1.3,1);
        \draw[dashed, black] (1.3,1) -- (1.6,1);
        \node[above left] at (1.1,1) {\tiny B};
        \node[above right] at (1.5,1) {\tiny $C^2$};
    \end{tikzpicture}
    \textbf{$+$}
    \begin{tikzpicture}[scale=1.5, line width=2pt]
        \draw[orange] (1,1) -- (1.6,1);
        
        \node[above left] at (1.1,1) {\tiny A};
        \node[above right] at (1.5,1) {\tiny $B^2$};
    \end{tikzpicture}
    \textit{на 2 \cdot}
    \begin{tikzpicture}[scale=1.5, line width=2pt]
        \draw[black] (1,1) -- (1.3,1);
        \draw[dashed, black] (1.3,1) -- (1.6,1);
        \node[above left] at (1.1,1) {\tiny B};
        \node[above right] at (1.5,1) {\tiny C};
    \end{tikzpicture}
    \cdot
    \begin{tikzpicture}[scale=1.5, line width=2pt]
        \draw[black] (1,1) -- (1.3,1);
        \node[above left] at (1.1,1) {\tiny B};
        \node[above right] at (1.2,1) {\tiny D};
    \end{tikzpicture}
    .



    \noindent \textit{Второй случай}
    
    \begin{tikzpicture}[scale=1.5, line width=2pt]
        \draw[blue] (1,1) -- (1.6,1);
        
        \node[above left] at (1.1,1) {\tiny E};
        \node[above right] at (1.5,1) {\tiny $G^2$};
    \end{tikzpicture}
    $<$
    \begin{tikzpicture}[scale=1.5, line width=2pt]
        \draw[orange] (1,1) -- (1.5,1);
        \node[above left] at (1.1,1) {\tiny E};
        \node[above right] at (1.4,1) {\tiny $F^2$};
    \end{tikzpicture}
    \textbf{$+$}
    \begin{tikzpicture}[scale=1.5, line width=2pt]
        \draw[black] (1,1) -- (1.4,1);
        
        \node[above left] at (1.1,1) {\tiny F};
        \node[above right] at (1.3,1) {\tiny $G^2$};
    \end{tikzpicture}
    \textit{на 2 \cdot}
    \begin{tikzpicture}[scale=1.5, line width=2pt]
        \draw[black] (1,1) -- (1.3,1);
        \node[above left] at (1.1,1) {\tiny F};
        \node[above right] at (1.2,1) {\tiny G};
    \end{tikzpicture}
    \cdot
    \begin{tikzpicture}[scale=1.5, line width=2pt]
        \draw[black] (1,1) -- (1.3,1);
        \draw[dashed, black] (1.3,1) -- (1.6,1);
        \node[above left] at (1.1,1) {\tiny F};
        \node[above right] at (1.5,1) {\tiny H};
    \end{tikzpicture}
    .
    \vspace{90mm}
    


\end{minipage}
\hfill
\begin{minipage}{0.3\textwidth}
\begin{center}
 \Large Первый случай
\end{center}
\begin{tikzpicture}[scale=1.5, line width=2pt]
    \draw[yellow] (1.1,9) -- (1.1,12);
    \draw[red] (0,9) -- (1.1,12);
    \draw[blue] (1.1,12) -- (3,9);
    \draw[dashed, black] (3,9) -- (1.1,9);
    \draw[black] (1.1,9) -- (0,9);
    

    \node[below left] at (0,9) {B};
    \node[below] at (1.1,9) {D};
    \node[above] at (1.1,12) {A};
    \node[below right] at (3,9) {C};
\end{tikzpicture}

\begin{center}
 \Large Второй случай
\end{center}
\begin{tikzpicture}[scale=1.5, line width=2pt]
    \draw[blue] (1.1,9) -- (3,12);
    \draw[yellow] (3,9) -- (3,12);
    \draw[red] (0,9) -- (3,12);
    \draw[dashed, black] (3,9) -- (1.1,9);
    \draw[black] (1.1,9) -- (0,9);
    

    \node[below left] at (0,9) {F};
    \node[below] at (1.1,9) {G};
    \node[above] at (3,12) {E};
    \node[below right] at (3,9) {H};
\end{tikzpicture}
\end{minipage}
	


\end{document}
