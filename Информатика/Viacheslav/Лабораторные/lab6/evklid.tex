\documentclass[a4paper, 10pt]{article}

\usepackage[russian]{babel}
\usepackage[T2A]{fontenc}
\usepackage[utf8]{inputenc}
\usepackage{multicol}
\usepackage{adjustbox}
\usepackage{amsmath}
\usepackage{tikz}
\usepackage{amssymb}



\usepackage{geometry}
\geometry{top=10mm}
\geometry{bottom=5mm}
\geometry{left=10mm}
\geometry{right=10mm}

\usepackage{graphicx}
\usepackage{wrapfig}

\pagestyle{empty}

\newcommand{\solidline}[6]{
    \begin{tikzpicture}[scale=1.5, line width=2pt]
        \draw[#6, #3] (#4,1) -- (#5,1);
        
        \node[above] at (#4,1) {\tiny #1};
        \node[above] at (#5,1) {\tiny #2};
    \end{tikzpicture}
}

\newcommand{\longline}[3]{
    \solidline{#1}{#2}{#3}{1}{1.6}{}
}

\newcommand{\shortline}[3]{
    \solidline{#1}{#2}{#3}{1}{1.4}{}
}

\newcommand{\shortdashedline}[3]{
    \solidline{#1}{#2}{#3}{1}{1.4}{dashed}
}

\newcommand{\splitline}[2]{
    \begin{tikzpicture}[scale=1.5, line width=2pt]
        \draw[black] (1,1) -- (1.3,1);
        \draw[dashed, black] (1.3,1) -- (1.6,1);
        \node[above left] at (1.1,1) {\tiny #1};
        \node[above right] at (1.6,1) {\tiny #2};
    \end{tikzpicture}
}



\begin{document}

\begin{minipage}{0.6\textwidth}
\hfill КНИГА II ПРЕДЛ. XIII. ТЕОРЕМА \hfill \llap{89}
\vspace{6mm}
    \begin{wrapfigure}{l}{0.13\textwidth}
        \includegraphics[width=0.14\textwidth]{image}
    \end{wrapfigure}
    \\
    \large \textit{о всяком треугольнике квадрат стороны, стягивающей острый угол, меньше суммы квадратов сторон, содержащих этот угол, на дважды прямоугольник, заключенный между любой из этих сторон и отрезком, отсекеаемым перпендикуляром из противоположного угла от этого отрезк или от продленного отрезка.}

    \noindent \textit{Первый случай} \\
    \longline{C}{$A^2$}{blue}$<$\splitline{B}{$C^2$}$+$\longline{A}{$B^2$}{orange}
    \textit{на 2} \cdot 
    \splitline{B}{C}
    \cdot
    \shortline{B}{D}{black}
    .
    
    \noindent \textit{Второй случай}
    
    \longline{B}{$C^2$}{blue}
    $<$
    \longline{B}{$F^2$}{orange}
    \textbf{$+$}
    \shortline{F}{$G^2$}{black}
    \textit{на 2} \cdot
    \shortline{F}{G}{black}
    \cdot
    \splitline{F}{H}
    .\\

    \begin{center}
        \large Предположим, перпендикуляр падает \\ внутри треугольника, тогда (пр. {\MakeUppercase{\romannumeral 2}}.\raisebox{-0.2em}{7})

        \splitline{B}{$C^2$}$+$\shortline{B}{$D^2$}{black}
        = 2 \cdot
        \splitline{B}{C} \cdot \shortline{B}{D}{black} + \shortdashedline{D}{$C^2$}{black}
    

    Теперь предположим, что перпендикуляр \\ падает вовне треугольника, тогда (пр. II.7)
    
    
    \end{center}
    

    \begin{flushright}
        ч.т.д.
    \end{flushright}
    \vspace{90mm}
    


\end{minipage}
\hfill
\begin{minipage}{0.3\textwidth}
    \begin{center}
     \Large Первый случай
    \end{center}
    \begin{tikzpicture}[scale=1.5, line width=2pt]
        \draw[yellow] (1.1,9) -- (1.1,12);
        \draw[red] (0,9) -- (1.1,12);
        \draw[blue] (1.1,12) -- (3,9);
        \draw[dashed, black] (3,9) -- (1.1,9);
        \draw[black] (1.1,9) -- (0,9);
        
    
        \node[below left] at (0,9) {B};
        \node[below] at (1.1,9) {D};
        \node[above] at (1.1,12) {A};
        \node[below right] at (3,9) {C};
    \end{tikzpicture}
    
    \begin{center}
     \Large Второй случай
    \end{center}
    \begin{tikzpicture}[scale=1.5, line width=2pt]
        \draw[blue] (1.1,9) -- (3,12);
        \draw[yellow] (3,9) -- (3,12);
        \draw[red] (0,9) -- (3,12);
        \draw[dashed, black] (3,9) -- (1.1,9);
        \draw[black] (1.1,9) -- (0,9);
        
    
        \node[below left] at (0,9) {F};
        \node[below] at (1.1,9) {G};
        \node[above] at (3,12) {E};
        \node[below right] at (3,9) {H};
    \end{tikzpicture}
\end{minipage}
	


\end{document}
